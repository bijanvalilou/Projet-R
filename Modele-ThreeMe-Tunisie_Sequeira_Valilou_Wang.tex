% Options for packages loaded elsewhere
\PassOptionsToPackage{unicode}{hyperref}
\PassOptionsToPackage{hyphens}{url}
%
\documentclass[
]{article}
\title{Modèle ThreeMe Tunisie}
\usepackage{etoolbox}
\makeatletter
\providecommand{\subtitle}[1]{% add subtitle to \maketitle
  \apptocmd{\@title}{\par {\large #1 \par}}{}{}
}
\makeatother
\subtitle{Implications macroéconomiques d'une Stratégie Nationale Bas
Carbone pour un petit pays en développement : le cas de la Tunisie}
\author{Lucia SEQUEIRA - Bijan VALILOU - Jin WANG}
\date{Janvier 2021}

\usepackage{amsmath,amssymb}
\usepackage{lmodern}
\usepackage{iftex}
\ifPDFTeX
  \usepackage[T1]{fontenc}
  \usepackage[utf8]{inputenc}
  \usepackage{textcomp} % provide euro and other symbols
\else % if luatex or xetex
  \usepackage{unicode-math}
  \defaultfontfeatures{Scale=MatchLowercase}
  \defaultfontfeatures[\rmfamily]{Ligatures=TeX,Scale=1}
\fi
% Use upquote if available, for straight quotes in verbatim environments
\IfFileExists{upquote.sty}{\usepackage{upquote}}{}
\IfFileExists{microtype.sty}{% use microtype if available
  \usepackage[]{microtype}
  \UseMicrotypeSet[protrusion]{basicmath} % disable protrusion for tt fonts
}{}
\makeatletter
\@ifundefined{KOMAClassName}{% if non-KOMA class
  \IfFileExists{parskip.sty}{%
    \usepackage{parskip}
  }{% else
    \setlength{\parindent}{0pt}
    \setlength{\parskip}{6pt plus 2pt minus 1pt}}
}{% if KOMA class
  \KOMAoptions{parskip=half}}
\makeatother
\usepackage{xcolor}
\IfFileExists{xurl.sty}{\usepackage{xurl}}{} % add URL line breaks if available
\IfFileExists{bookmark.sty}{\usepackage{bookmark}}{\usepackage{hyperref}}
\hypersetup{
  pdftitle={Modèle ThreeMe Tunisie},
  pdfauthor={Lucia SEQUEIRA - Bijan VALILOU - Jin WANG},
  hidelinks,
  pdfcreator={LaTeX via pandoc}}
\urlstyle{same} % disable monospaced font for URLs
\usepackage[margin=1in]{geometry}
\usepackage{graphicx}
\makeatletter
\def\maxwidth{\ifdim\Gin@nat@width>\linewidth\linewidth\else\Gin@nat@width\fi}
\def\maxheight{\ifdim\Gin@nat@height>\textheight\textheight\else\Gin@nat@height\fi}
\makeatother
% Scale images if necessary, so that they will not overflow the page
% margins by default, and it is still possible to overwrite the defaults
% using explicit options in \includegraphics[width, height, ...]{}
\setkeys{Gin}{width=\maxwidth,height=\maxheight,keepaspectratio}
% Set default figure placement to htbp
\makeatletter
\def\fps@figure{htbp}
\makeatother
\setlength{\emergencystretch}{3em} % prevent overfull lines
\providecommand{\tightlist}{%
  \setlength{\itemsep}{0pt}\setlength{\parskip}{0pt}}
\setcounter{secnumdepth}{5}
\usepackage{fancyhdr}
\pagestyle{fancy}
\fancyfoot[CO,CE]{Modèle ThreeMe Tunisie}
\fancyfoot[LE,RO]{\thepage}
\ifLuaTeX
  \usepackage{selnolig}  % disable illegal ligatures
\fi
\usepackage[]{biblatex}
\addbibresource{references.bib}

\begin{document}
\maketitle

{
\setcounter{tocdepth}{2}
\tableofcontents
}
\newpage

\hypertarget{introduction}{%
\section{Introduction}\label{introduction}}

\hypertarget{literature-review---pas-urgent-on-le-fait-en-paralluxe8le}{%
\section{Literature review - pas urgent, on le fait en
parallèle}\label{literature-review---pas-urgent-on-le-fait-en-paralluxe8le}}

\hypertarget{energy-and-economy-framework-in-tunisia}{%
\subsection{Energy and economy framework in
Tunisia}\label{energy-and-economy-framework-in-tunisia}}

\hypertarget{climate-policy}{%
\subsection{Climate policy}\label{climate-policy}}

\hypertarget{carbon-tax}{%
\subsubsection{Carbon tax}\label{carbon-tax}}

\hypertarget{energetic-subsidy}{%
\subsubsection{Energetic subsidy}\label{energetic-subsidy}}

\hypertarget{methodology}{%
\section{Methodology}\label{methodology}}

\hypertarget{threeme-model}{%
\subsection{ThreeME model}\label{threeme-model}}

\hypertarget{a-model-of-neo-keynesian-inspiration}{%
\subsubsection{A model of neo-Keynesian
inspiration}\label{a-model-of-neo-keynesian-inspiration}}

\hypertarget{threeme-model-tunisia-adaptation-au-cas-dun-pays-en-duxe9veloppement}{%
\subsubsection{ThreeMe model Tunisia : Adaptation au cas d'un pays en
développement}\label{threeme-model-tunisia-adaptation-au-cas-dun-pays-en-duxe9veloppement}}

\hypertarget{description-of-scenario}{%
\subsection{Description of scenario}\label{description-of-scenario}}

\begin{itemize}
\tightlist
\item
  describe redistribution
\end{itemize}

\hypertarget{choice-of-evaluation-indicators}{%
\subsection{Choice of evaluation
indicators}\label{choice-of-evaluation-indicators}}

\hypertarget{data-structure}{%
\subsubsection{Data structure}\label{data-structure}}

\hypertarget{kaya-identity}{%
\subsubsection{Kaya identity}\label{kaya-identity}}

The Kaya identity is an identity where the total emission of carbon
dioxide can be explained by four product driving forcec as population,
Gross Domestic Product (GDP) per capita, enerny intensity over GDP and
carbon intensity over energy consumption. \textcite{gazàef2016}

\[ Emission_{CO_2} = POP \times\frac{GDP}{POP} \times \frac{TEC}{GDP} \times \frac{TEM}{TEC}\]

\hypertarget{aggregated-level}{%
\paragraph{Aggregated level}\label{aggregated-level}}

\hypertarget{sectorial-level}{%
\paragraph{Sectorial level}\label{sectorial-level}}

\hypertarget{results-et-discussions}{%
\section{Results et discussions}\label{results-et-discussions}}

\hypertarget{comparasion-of-redistribution}{%
\subsection{Comparasion of
redistribution}\label{comparasion-of-redistribution}}

\begin{itemize}
\tightlist
\item
  graphique PIB : grandes tendances des scénarii Taxe carbone, Levée des
  subventions énergétiques
\item
  chomage : grandes tendances des scénarii Taxe carbone, Levée des
  subventions énergétiques
\item
  émissions : grandes tendances des scénarii Taxe carbone, Levée des
  subventions énergétiques
\end{itemize}

\hypertarget{carbon-tax-1}{%
\subsection{Carbon tax}\label{carbon-tax-1}}

Avec/ sans recyclage Focus inter-sectoriel

\hypertarget{lifting-of-energy-subsidies}{%
\subsection{Lifting of energy
subsidies}\label{lifting-of-energy-subsidies}}

Avec/ sans recyclage Focus inter-sectoriel

\hypertarget{enr}{%
\subsection{ENR}\label{enr}}

\hypertarget{snbc}{%
\subsection{SNBC}\label{snbc}}

\hypertarget{prolongements}{%
\section{Prolongements}\label{prolongements}}

\hypertarget{dautres-leviers-pour-analyser-ouverture}{%
\subsection{D'autres leviers pour analyser
(ouverture)}\label{dautres-leviers-pour-analyser-ouverture}}

\hypertarget{lamuxe9lioration-du-cadre-statistique}{%
\subsection{L'amélioration du cadre
statistique}\label{lamuxe9lioration-du-cadre-statistique}}

OUverture : pb du secteur informel non pris en compte par la
comptabilité nationale

\hypertarget{conclusion}{%
\section{Conclusion}\label{conclusion}}

\printbibliography[title=Bibliographie]

\end{document}
