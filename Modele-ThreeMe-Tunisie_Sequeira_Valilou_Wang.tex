% Options for packages loaded elsewhere
\PassOptionsToPackage{unicode}{hyperref}
\PassOptionsToPackage{hyphens}{url}
%
\documentclass[
]{article}
\title{Modèle ThreeMe Tunisie}
\usepackage{etoolbox}
\makeatletter
\providecommand{\subtitle}[1]{% add subtitle to \maketitle
  \apptocmd{\@title}{\par {\large #1 \par}}{}{}
}
\makeatother
\subtitle{Implications macroéconomiques d'une Stratégie Nationale Bas
Carbone pour un petit pays en développement : le cas de la Tunisie}
\author{Lucia SEQUEIRA - Bijan VALILOU - Jin WANG}
\date{Janvier 2021}

\usepackage{amsmath,amssymb}
\usepackage{lmodern}
\usepackage{iftex}
\ifPDFTeX
  \usepackage[T1]{fontenc}
  \usepackage[utf8]{inputenc}
  \usepackage{textcomp} % provide euro and other symbols
\else % if luatex or xetex
  \usepackage{unicode-math}
  \defaultfontfeatures{Scale=MatchLowercase}
  \defaultfontfeatures[\rmfamily]{Ligatures=TeX,Scale=1}
\fi
% Use upquote if available, for straight quotes in verbatim environments
\IfFileExists{upquote.sty}{\usepackage{upquote}}{}
\IfFileExists{microtype.sty}{% use microtype if available
  \usepackage[]{microtype}
  \UseMicrotypeSet[protrusion]{basicmath} % disable protrusion for tt fonts
}{}
\makeatletter
\@ifundefined{KOMAClassName}{% if non-KOMA class
  \IfFileExists{parskip.sty}{%
    \usepackage{parskip}
  }{% else
    \setlength{\parindent}{0pt}
    \setlength{\parskip}{6pt plus 2pt minus 1pt}}
}{% if KOMA class
  \KOMAoptions{parskip=half}}
\makeatother
\usepackage{xcolor}
\IfFileExists{xurl.sty}{\usepackage{xurl}}{} % add URL line breaks if available
\IfFileExists{bookmark.sty}{\usepackage{bookmark}}{\usepackage{hyperref}}
\hypersetup{
  pdftitle={Modèle ThreeMe Tunisie},
  pdfauthor={Lucia SEQUEIRA - Bijan VALILOU - Jin WANG},
  hidelinks,
  pdfcreator={LaTeX via pandoc}}
\urlstyle{same} % disable monospaced font for URLs
\usepackage[margin=1in]{geometry}
\usepackage{graphicx}
\makeatletter
\def\maxwidth{\ifdim\Gin@nat@width>\linewidth\linewidth\else\Gin@nat@width\fi}
\def\maxheight{\ifdim\Gin@nat@height>\textheight\textheight\else\Gin@nat@height\fi}
\makeatother
% Scale images if necessary, so that they will not overflow the page
% margins by default, and it is still possible to overwrite the defaults
% using explicit options in \includegraphics[width, height, ...]{}
\setkeys{Gin}{width=\maxwidth,height=\maxheight,keepaspectratio}
% Set default figure placement to htbp
\makeatletter
\def\fps@figure{htbp}
\makeatother
\setlength{\emergencystretch}{3em} % prevent overfull lines
\providecommand{\tightlist}{%
  \setlength{\itemsep}{0pt}\setlength{\parskip}{0pt}}
\setcounter{secnumdepth}{5}
\usepackage{fancyhdr}
\pagestyle{fancy}
\fancyfoot[CO,CE]{Modèle ThreeMe Tunisie}
\fancyfoot[LE,RO]{\thepage}
\ifLuaTeX
  \usepackage{selnolig}  % disable illegal ligatures
\fi
\usepackage[]{biblatex}

\begin{document}
\maketitle

{
\setcounter{tocdepth}{2}
\tableofcontents
}
\textbackslash begin\{document\}

\newpage

\hypertarget{introduction}{%
\section{Introduction}\label{introduction}}

\hypertarget{revue-de-la-littuxe9rature}{%
\section{Revue de la littérature}\label{revue-de-la-littuxe9rature}}

\hypertarget{cadre-uxe9nerguxe9tique-et-uxe9conomique-en-tunisie}{%
\subsection{Cadre énergétique et économique en
Tunisie}\label{cadre-uxe9nerguxe9tique-et-uxe9conomique-en-tunisie}}

\hypertarget{moduxe8le-threeme}{%
\subsection{Modèle ThreeME}\label{moduxe8le-threeme}}

\hypertarget{un-moduxe8le-dinspiration-nuxe9o-keynuxe9sienne}{%
\subsubsection{Un modèle d'inspiration
néo-keynésienne}\label{un-moduxe8le-dinspiration-nuxe9o-keynuxe9sienne}}

\hypertarget{moduxe8le-threeme-tunisie-adaptation-au-cas-dun-pays-en-duxe9veloppement}{%
\subsubsection{Modèle ThreeMe Tunisie : Adaptation au cas d'un pays en
développement}\label{moduxe8le-threeme-tunisie-adaptation-au-cas-dun-pays-en-duxe9veloppement}}

\hypertarget{politiques-clilmatiques}{%
\subsection{Politiques clilmatiques}\label{politiques-clilmatiques}}

\hypertarget{taxe-carbone}{%
\subsubsection{Taxe carbone}\label{taxe-carbone}}

\hypertarget{subvention-uxe9nerguxe9tique}{%
\subsubsection{Subvention
énergétique}\label{subvention-uxe9nerguxe9tique}}

\hypertarget{muxe9thodologie}{%
\section{Méthodologie}\label{muxe9thodologie}}

\hypertarget{descriptifs-des-scuxe9narii-uxe0-analyser}{%
\subsection{Descriptifs des scénarii à
analyser}\label{descriptifs-des-scuxe9narii-uxe0-analyser}}

\hypertarget{choix-des-indicateurs-duxe9valuation}{%
\subsection{Choix des indicateurs
d'évaluation}\label{choix-des-indicateurs-duxe9valuation}}

\begin{itemize}
\item
  Présentation des tables de données :
\item
  Identité de Kaya : décomposition sectorielle
\end{itemize}

\hypertarget{ruxe9sultats-et-discussions}{%
\section{Résultats et discussions}\label{ruxe9sultats-et-discussions}}

\hypertarget{introduction-1}{%
\subsection{Introduction}\label{introduction-1}}

\begin{itemize}
\tightlist
\item
  graphique PIB : grandes tendances des scénarii Taxe carbone, Levée des
  subventions énergétiques
\item
  chomage : grandes tendances des scénarii Taxe carbone, Levée des
  subventions énergétiques
\item
  émissions : grandes tendances des scénarii Taxe carbone, Levée des
  subventions énergétiques
\end{itemize}

\hypertarget{taxe-carbone-1}{%
\subsection{Taxe carbone}\label{taxe-carbone-1}}

Avec/ sans recyclage Focus inter-sectoriel

\hypertarget{levuxe9e-des-subventions-uxe9nerguxe9tiques}{%
\subsection{Levée des subventions
énergétiques}\label{levuxe9e-des-subventions-uxe9nerguxe9tiques}}

Avec/ sans recyclage Focus inter-sectoriel

\hypertarget{enr}{%
\subsection{ENR}\label{enr}}

\hypertarget{snbc}{%
\subsection{SNBC}\label{snbc}}

\hypertarget{prolongements}{%
\section{Prolongements}\label{prolongements}}

\hypertarget{dautres-leviers-pour-analyser-ouverture}{%
\subsection{D'autres leviers pour analyser
(ouverture)}\label{dautres-leviers-pour-analyser-ouverture}}

\hypertarget{lamuxe9lioration-du-cadre-statistique}{%
\subsection{L'amélioration du cadre
statistique}\label{lamuxe9lioration-du-cadre-statistique}}

OUverture : pb du secteur informel non pris en compte par la
comptabilité nationale

\hypertarget{conclusion}{%
\section{Conclusion}\label{conclusion}}

\printbibliography[title=Bibliographie]

\end{document}
